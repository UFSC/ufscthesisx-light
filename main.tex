%https://github.com/SublimeText/LaTeXTools/issues/1439
%!TEX output_directory=latexcache

% ------------------------------------------------------------------------
% ------------------------------------------------------------------------
% Modelo UFSC para Trabalhos Academicos (tese de doutorado, dissertação de
% mestrado) utilizando a classe abntex2
%
% Autor: Alisson Lopes Furlani
%   Modificações:
%   - 27/08/2019: Alisson L. Furlani, add pacote 'glossaries' para listas
% ------------------------------------------------------------------------
% ------------------------------------------------------------------------

% https://tex.stackexchange.com/questions/516058/why-my-biblatex-language-is-not-changing-when-passing-the-language-on-my-documen
% https://tex.stackexchange.com/questions/385895/how-to-make-passoptionstopackage-add-the-option-as-the-last
% \PassOptionsToPackage{brazil,main=english,spanish,french}{babel}
\PassOptionsToPackage{main=brazil,english,spanish,french}{babel}

\documentclass[
    % -- opções da classe memoir --
    12pt, % tamanho da fonte
    %openright, % capítulos começam em pág ímpar (insere página vazia caso preciso)
    oneside, % para impressão no anverso. Oposto a twoside
    a4paper, % tamanho do papel.
    % -- opções da classe abntex2 --
    chapter=TITLE, % títulos de capítulos convertidos em letras maiúsculas
    section=TITLE, % títulos de seções convertidos em letras maiúsculas
    %subsection=TITLE, % títulos de subseções convertidos em letras maiúsculas
    %subsubsection=TITLE, % títulos de subsubseções convertidos em letras maiúsculas
    % -- opções do pacote babel --
    english, % idioma adicional para hifenização
    %french, % idioma adicional para hifenização
    %spanish, % idioma adicional para hifenização
    brazil, % https://tex.stackexchange.com/questions/484400/changing-the-cleveref-package-language-conjunction-definition
    brazilian, % o último idioma é o principal do documento?
    ]{setup/ufscthesisx}

\addbibresource{aftertext/references.bib} % Seus arquivos de referências

% ---
% Informações de dados para CAPA e FOLHA DE ROSTO
% ---
% FIXME Substituir 'Nome completo do autor' pelo seu nome.
\autor{Nome completo do autor(a)}
% FIXME Substituir 'Título do trabalho' pelo título da trabalho.
\titulo{Título do trabalho}
% FIXME Substituir 'Subtítulo (se houver)' pelo subtítulo da trabalho.
% Caso não tenha substítulo, comente a linha a seguir.
\subtitulo{Subtítulo (se houver)}
% FIXME Substituir 'XXXXXX' pelo nome do seu
% orientador.
\orientador{Prof. XXXXXX, Dr.}
% FIXME Se for orientado por uma mulher, comente a linha acima e descomente a linha a seguir.
% \orientador[Orientadora]{Nome da orientadora, Dra.}
% FIXME Substituir 'XXXXXX' pelo nome do seu
% coorientador. Caso não tenha coorientador, comente a linha a seguir.
\coorientador{Prof. XXXXXX, Dr.}
% FIXME Se for coorientado por uma mulher, comente a linha acima e descomente a linha a seguir.
% \coorientador[Coorientadora]{XXXXXX, Dra.}
% FIXME Substituir 'XXXXXX' pelo nome do Coordenador do
% programa/curso.
\coordenador{Prof. XXXXXX, Dr.}
% FIXME Se for coordenadora mulher, comente a linha acima e descomente a linha a seguir.
% \coordenador[Coordenadora]{Nome da Coordenadora, Dra.}
% FIXME Substituir '[ano]' pelo ano (ano) em que seu trabalho foi defendido.
\ano{ano}
% FIXME Substituir '[dia] de [mês] de [ano]' pela data em que ocorreu sua defesa.
\data{[dia] de [mês] de [ano]}
% FIXME Substituir 'Local' pela cidade em que ocorreu sua defesa.
\local{Local}
\instituicaosigla{UFSC}
\instituicao{Universidade Federal de Santa Catarina}
% FIXME Substituir 'Dissertação/Tese' pelo tipo de trabalho (Tese, Dissertação).
\tipotrabalho{Dissertação/Tese}
% FIXME Substituir '[mestre/doutor] em XXXXXX' pela grau adequado.
\formacao{[mestre/doutor] em [nome do título obtido pelo Programa]}
% FIXME Substituir '[mestrado/doutorado]' pelo nivel adequado.
\nivel{[mestrado/doutorado]}
% FIXME Substituir 'Programa de Pós-Graduação em XXXXXX' pela curso adequado.
\programa{Programa de Pós-Graduação em XXXXXX}
% FIXME Substituir 'Campus XXXXXX ou Centro de XXXXXX' pelo campus ou centro adequado.
\centro{Campus XXXXXX ou Centro de XXXXXX}
% FIXME: Preencha com Campus XXXXXX     ou     Centro de XXXXXX
\campus{Campus Reitor João David Ferreira Lima}
\preambulo
{%
\imprimirtipotrabalho~submetida~ao~\imprimirprograma~da~\imprimirinstituicao~para~a~obtenção~do~título~de~\imprimirformacao.
}
% ---

% ---
% Configurações de aparência do PDF final
% ---
% alterando o aspecto da cor azul
\definecolor{blue}{RGB}{41,5,195}
% informações do PDF
\makeatletter
\hypersetup{
        %pagebackref=true,
        pdftitle={\@title},
        pdfauthor={\@author},
        pdfsubject={\imprimirpreambulo},
        pdfcreator={LaTeX with abnTeX2},
        pdfkeywords={ufsc, latex, abntex2},
        colorlinks=true,            % false: boxed links; true: colored links
        linkcolor=black,%blue,              % color of internal links
        citecolor=black,%blue,              % color of links to bibliography
        filecolor=black,%magenta,           % color of file links
        urlcolor=black,%blue,
        bookmarksdepth=4
}
\makeatother
% ---

% ---
% compila a lista de abreviaturas e siglas e a lista de símbolos
% ---

% Declaração das siglas
\siglalista{ABNT}{Associação Brasileira de Normas Técnicas}

% Declaração dos simbolos
\simbololista{C}{\ensuremath{C}}{Circunferência de um círculo}
\simbololista{pi}{\ensuremath{\pi}}{Número pi}
\simbololista{r}{\ensuremath{r}}{Raio de um círculo}
\simbololista{A}{\ensuremath{A}}{Área de um círculo}

% compila a lista de abreviaturas e siglas e a lista de símbolos
\makenoidxglossaries

% ---

% ---
% compila o indice
% ---
\makeindex
% ---

% ----
% Início do documento
% ----
\begin{document}

% Seleciona o idioma do documento (conforme pacotes do babel)
%\selectlanguage{english}
\selectlanguage{brazil}

% Retira espaço extra obsoleto entre as frases.
\frenchspacing

% Espaçamento 1.5 entre linhas
\OnehalfSpacing

% Corrige justificação
%\sloppy

% ----------------------------------------------------------
% ELEMENTOS PRÉ-TEXTUAIS
% ----------------------------------------------------------
% \pretextual %a macro \pretextual é acionado automaticamente no início de \begin{document}
% ---
% Capa, folha de rosto, ficha bibliografica, errata, folha de apróvação
% Dedicatória, agradecimentos, epígrafe, resumos, listas
% ---
% ---
% Capa
% ---
% https://tex.stackexchange.com/questions/386446/how-to-fix-destination-with-the-same-identifier-namepage-a-has-been-already
% https://tex.stackexchange.com/questions/67989/pdftex-warning-has-been-referenced-but-does-not-exist-replaced-by-a-fixed-one
\hypersetup{pageanchor=false}
\PRIVATEbookmarkthis{Capa}
\imprimircapa
\hypersetup{pageanchor=true}
% ---

% ---
% Folha de rosto
% (o * indica que haverá a ficha bibliográfica)
% ---
\imprimirfolhaderosto*
% ---

% ---
% Inserir a ficha bibliografica
% ---
% http://ficha.bu.ufsc.br/
\begin{fichacatalografica}
	\includepdf{beforetext/Ficha_Catalografica.pdf}
\end{fichacatalografica}
% ---

% ---
% Inserir folha de aprovação
% ---
\begin{folhadeaprovacao}
	\OnehalfSpacing
	\centering
	\imprimirautor\\%
	\vspace*{10pt}
	\textbf{\imprimirtitulo}%
	\ifnotempty{\imprimirsubtitulo}{:~\imprimirsubtitulo}\\%
	%		\vspace*{31.5pt}%3\baselineskip
	\vspace*{\baselineskip}
	%\begin{minipage}{\textwidth}
	O presente trabalho em nível de \imprimirnivel~foi avaliado e aprovado por banca examinadora composta pelos seguintes membros:\\
	%\end{minipage}%
	\vspace*{\baselineskip}
	Prof.(a) xxxx, Dr(a).\\
	Universidade xxxx\\
	\vspace*{\baselineskip}
	Prof.(a) xxxx, Dr(a).\\
	Universidade xxxx\\
	\vspace*{\baselineskip}
	Prof.(a) xxxx, Dr(a).\\
	Universidade xxxx\\
	\vspace*{2\baselineskip}
	\begin{minipage}{\textwidth}
		Certificamos que esta é a \textbf{versão original e final} do trabalho de conclusão que foi julgado adequado para obtenção do título de \imprimirformacao.\\
	\end{minipage}
	%    \vspace{-0.7cm}
	\vspace*{\fill}
	\assinatura{\OnehalfSpacing\imprimircoordenador \\ \imprimircoordenadorRotulo~do Programa}
	\vspace*{\fill}
	\assinatura{\OnehalfSpacing\imprimirorientador \\ \imprimirorientadorRotulo}
	%	\ifnotempty{\imprimircoorientador}{
	%	\assinatura{\imprimircoorientador \\ \imprimircoorientadorRotulo \\
	%		\imprimirinstituicao~--~\imprimirinstituicaosigla}
	%	}
	% \newpage
	\vspace*{\fill}
	\imprimirlocal, \imprimirdata.
\end{folhadeaprovacao}
% ---

% ---
% Dedicatória
% ---
\begin{dedicatoria}
	\vspace*{\fill}
	\noindent
	\begin{adjustwidth*}{}{5.5cm}
		Este trabalho é dedicado aos meus colegas de classe e aos meus queridos pais.
	\end{adjustwidth*}
\end{dedicatoria}
% ---

% ---
% Agradecimentos
% ---
\begin{agradecimentos}
	Inserir os agradecimentos aos colaboradores à execução do trabalho.

	Xxxxxxxxxxxxxxxxxxxxxxxxxxxxxxxxxxxxxxxxxxxxxxxxxxxxxxxxxxxxxxxxxxxxxx.
\end{agradecimentos}
% ---

% ---
% Epígrafe
% ---
\begin{epigrafe}
	\vspace*{\fill}
	\begin{flushright}
		\textit{``Texto da Epígrafe.\\
			Citação relativa ao tema do trabalho.\\
			É opcional. A epígrafe pode também aparecer\\
			na abertura de cada seção ou capítulo.\\
			Deve ser elaborada de acordo com a NBR 10520.''\\
			(Autor da epígrafe, ano)}
	\end{flushright}
\end{epigrafe}
% ---

% ---
% RESUMOS
% ---

% resumo em português
\setlength{\absparsep}{18pt} % ajusta o espaçamento dos parágrafos do resumo
\begin{resumo}
	\SingleSpacing
	No resumo são ressaltados o objetivo da pesquisa, o método utilizado, as discussões e os resultados com destaque apenas para os pontos principais. O resumo deve ser significativo, composto de uma sequência de frases concisas, afirmativas, e não de uma enumeração de tópicos. Não deve conter citações. Deve usar o verbo na voz ativa e na terceira pessoa do singular. O texto do resumo deve ser digitado, em um único bloco, sem espaço de parágrafo. O espaçamento entre linhas é simples e o tamanho da fonte é 12. Abaixo do resumo, informar as palavras-chave (palavras ou expressões significativas retiradas do texto) ou, termos retirados de thesaurus da área. Deve conter de 150 a 500 palavras. O resumo é elaborado de acordo com a NBR 6028.

	\parOriginal\vspace{\onelineskip}
	\textbf{Palavras-chave}: Palavra-chave 1. Palavra-chave 2. Palavra-chave 3.
\end{resumo}

% resumo em inglês
\begin{resumo}[Abstract]
	\SingleSpacing
	\begin{otherlanguage*}{english}
		Resumo traduzido para outros idiomas, neste caso, inglês. Segue o formato do resumo feito na língua vernácula. As palavras-chave traduzidas, versão em língua estrangeira, são colocadas abaixo do texto precedidas pela expressão “Keywords”, separadas por ponto.

	\parOriginal\vspace{\onelineskip}
	\textbf{Keywords}: Keyword 1. Keyword 2. Keyword 3.
	\end{otherlanguage*}
\end{resumo}

%% resumo em francês
%\begin{resumo}[Résumé]
% \begin{otherlanguage*}{french}
%    Il s'agit d'un résumé en français.
%
%   \textbf{Mots-clés}: latex. abntex. publication de textes.
% \end{otherlanguage*}
%\end{resumo}
%
%% resumo em espanhol
%\begin{resumo}[Resumen]
% \begin{otherlanguage*}{spanish}
%   Este es el resumen en español.
%
%   \textbf{Palabras clave}: latex. abntex. publicación de textos.
% \end{otherlanguage*}
%\end{resumo}
%% ---

{%hidelinks
	\hypersetup{hidelinks}
	% ---
	% inserir lista de ilustrações
	% ---
	\pdfbookmark[0]{\listfigurename}{lof}
	\listoffigures*
	\cleardoublepage
	% ---

	% ---
	% inserir lista de quadros
	% ---
	\pdfbookmark[0]{\listofquadrosname}{loq}
	\listofquadros*
	\cleardoublepage
	% ---

	% ---
	% inserir lista de tabelas
	% ---
	\pdfbookmark[0]{\listtablename}{lot}
	\listoftables*
	\cleardoublepage
	% ---

	% ---
	% inserir lista de abreviaturas e siglas (devem ser declarados no preambulo)
	% ---
	\imprimirlistadesiglas
	% ---

	% ---
	% inserir lista de símbolos (devem ser declarados no preambulo)
	% ---
	\imprimirlistadesimbolos
	% ---

	% ---
	% inserir o sumario
	% ---
	\pdfbookmark[0]{\contentsname}{toc}
	\tableofcontents*
	\cleardoublepage

}%hidelinks
% ---

% ---

% ----------------------------------------------------------
% ELEMENTOS TEXTUAIS
% ----------------------------------------------------------
\textual

% ---
% 1 - Introdução
% ---
% ----------------------------------------------------------
\chapter{Introdução}
% ----------------------------------------------------------

As orientações aqui apresentadas são baseadas em um conjunto de normas elaboradas pela \gls{ABNT}. Além das normas técnicas, a Biblioteca também elaborou uma série de tutoriais, guias, \textit{templates} os quais estão disponíveis em seu site, no endereço \url{http://portal.bu.ufsc.br/normalizacao/}.

Paralelamente ao uso deste \textit{template} recomenda-se que seja utilizado o \textbf{Tutorial de Trabalhos Acadêmicos} (disponível neste link \url{https://repositorio.ufsc.br/handle/123456789/180829}) e/ou que o discente \textbf{participe das capacitações oferecidas da Biblioteca Universitária da UFSC}.

Este \textit{template} está configurado apenas para a impressão utilizando o anverso das folhas, caso você queira imprimir usando a frente e o verso, acrescente a opção \textit{openright} e mude de \textit{oneside} para \textit{twoside} nas configurações da classe \textit{abntex2} no início do arquivo principal \textit{main.tex} \cite{abntex2classe}.

Conforme a \href{https://repositorio.ufsc.br/bitstream/handle/123456789/197121/RN46.2019.pdf?sequence=1&isAllowed=y}{Resolução NORMATIVA nº 46/2019/CPG} as dissertações e teses não serão mais entregues em formato impresso na Biblioteca Universitária. Consulte o Repositório Institucional da UFSC ou sua Secretaria de Pós Graduação sobre os procedimentos para a entrega.

\nocite{NBR6023:2002}
\nocite{NBR6027:2012}
\nocite{NBR6028:2003}
\nocite{NBR10520:2002}

% ----------------------------------------------------------
\section{Objetivos}
% ----------------------------------------------------------

Nas seções abaixo estão descritos o objetivo geral e os objetivos
específicos.

% ----------------------------------------------------------
\subsection{Objetivo Geral}
% ----------------------------------------------------------

Descrição...

% ----------------------------------------------------------
\subsection{Objetivos Específicos}
% ----------------------------------------------------------

Descrição...

% ---

% ---
% 2 - Capítulo 2
% ---
% ----------------------------------------------------------
\chapter{Desenvolvimento}\label{cap:desenvolvimento}
% ----------------------------------------------------------
Deve-se inserir texto entre as seções.

% ----------------------------------------------------------
\section{Exposição do tema ou matéria}
% ----------------------------------------------------------

É a parte principal e mais extensa do trabalho. Deve apresentar a fundamentação teórica, a metodologia, os resultados e a discussão. Divide-se em seções e subseções conforme a NBR 6024 \cite{NBR6024:2012}.

Quanto à sua estrutura e projeto gráfico, segue as recomendações da \gls{ABNT} para preparação de trabalhos acadêmicos, a NBR 14724, de 2011 \cite{NBR14724:2011}.

\begin{figure}[htb]
	\caption{\label{fig:Fig_1}Elementos do trabalho acadêmico.}
	\begin{center}
		\includegraphics{pictures/imagem.pdf}
	\end{center}
	\fonte{Universidade Federal do Paraná (1996).}
\end{figure}

% ----------------------------------------------------------
\subsection{Formatação do texto}
% ----------------------------------------------------------

No que diz respeito à estrutura do trabalho, recomenda-se que:
\begin{alineas}
	\item o texto deve ser justificado, digitado em cor preta, podendo utilizar outras cores somente para as ilustrações;
	\item utilizar papel branco ou reciclado para impressão;
	\item os elementos pré-textuais devem iniciar no anverso da folha, com exceção da ficha catalográfica ou ficha de identificação da obra;
	\item os elementos textuais e pós-textuais devem ser digitados no anverso e verso das folhas, quando o trabalho for impresso. As seções primárias devem começar sempre em páginas ímpares, quando o trabalho for impresso. Deixar um espaço entre o título da seção/subseção e o texto e entre o texto e o título da subseção.
\end{alineas}

No \autoref{qua:Quadro_1} estão as especificações para a formatação do texto.

\begin{quadro}[htb]
	\centering
	\caption{\label{qua:Quadro_1}Formatação do texto.}
	\begin{tabular}{|l|p{11cm}|}
		\hline
		\textbf{Formato do papel} & A4.\\ \hline
		\textbf{Impressão}        & A norma recomenda que caso seja necessário imprimir, deve-se utilizar a frente e o verso da página.\\ \hline
		\textbf{Margens}          & Superior: 3, Inferior: 2, Interna: 3 e Externa: 2. Usar margens espelhadas quando o  trabalho for impresso.\\ \hline
		\textbf{Paginação}        & As páginas dos elementos pré-textuais devem ser contadas, mas não numeradas. Para trabalhos digitados somente no anverso, a numeração das páginas deve constar no canto superior direito da página, a 2 cm da borda, figurando a partir da primeira folha da  parte textual. Para trabalhos digitados no anverso e no verso, a numeração deve constar no canto superior direito, no anverso, e no canto superior esquerdo no verso.\\ \hline
		\textbf{Espaçamento}      & O texto deve ser redigido com espaçamento entre linhas 1,5, excetuando-se as citações de mais de três linhas, notas de rodapé, referências, legendas das ilustrações e das tabelas, natureza (tipo do trabalho, objetivo, nome da instituição a que é submetido e área de concentração), que devem ser digitados em espaço simples, com fonte menor. As referências devem ser separadas entre si por um espaço simples em branco.\\ \hline
		\textbf{Paginação}        & A contagem inicia na folha de rosto, mas se insere o número da página na introdução até o final do trabalho.\\ \hline
		\textbf{Fontes sugeridas} & Arial ou Times New Roman.\\ \hline
		\textbf{Tamanho da fonte} & \textbf{Fonte tamanho 12 para o texto}, incluindo os títulos das seções e subseções. As citações com mais de três linhas, notas de rodapé, paginação, dados internacionais de catalogação, legendas e fontes das ilustrações e das tabelas devem ser de tamanho menor. Adotamos, neste \textit{template} \textbf{fonte tamanho 10}.\\ \hline
		\textbf{Nota de rodapé}   & Devem ser digitadas dentro da margem, ficando separadas por um espaço simples por entre as linhas e por filete de 5 cm a partir da margem esquerda. A partir da segunda linha, devem ser alinhadas embaixo da primeira letra da primeira palavra da primeira linha.\\ \hline
	\end{tabular}
	\fonte{\textcite{NBR14724:2011}.}
\end{quadro}

% ----------------------------------------------------------
\subsubsection{As ilustrações}
% ----------------------------------------------------------

Independentemente do tipo de ilustração (quadro, desenho, figura, fotografia, mapa, entre outros), a sua identificação aparece na parte superior, precedida da palavra designativa.

\begin{citacao}
	Após a ilustração, na parte inferior, indicar a fonte consultada (elemento obrigatório, mesmo que seja produção do próprio autor), legenda, notas e outras informações necessárias à sua compreensão (se houver). A ilustração deve ser citada no texto e inserida o mais próximo possível do texto a que se refere. \cite[p. 11]{NBR14724:2011}.
\end{citacao}

% ----------------------------------------------------------
\subsubsection{Equações e fórmulas}
% ----------------------------------------------------------

As equações e fórmulas devem ser destacadas no texto para facilitar a leitura.  Para numerá-las, usar algarismos arábicos entre parênteses e alinhados à direita. Pode-se adotar uma entrelinha maior do que a usada no texto \cite{NBR14724:2011}.

Exemplos, \typeref{eq:Eq_1,eq:Eq_2}.

\begin{equation}\label{eq:Eq_1}
\gls{C} = 2 \gls{pi} \gls{r}
\end{equation}

\begin{equation}\label{eq:Eq_2}
\gls{A} = \gls{pi} \gls{r}^2
\end{equation}

% ----------------------------------------------------------
\subsubsubsection{Exemplo tabela}
% ----------------------------------------------------------

De acordo com \textcite{ibge1993}, tabela é uma forma não discursiva de apresentar informações em que os números representam a informação central. Ver \autoref{tab:Tab_1}.

\begin{table}[htb]
	\ABNTEXfontereduzida
	\caption{\label{tab:Tab_1}Médias concentrações urbanas 2010-2011.}
	\begin{tabular}{@{}p{3.0cm}p{1.5cm}p{2cm}p{2.5cm}p{2.5cm}p{2.5cm}@{}}
		\toprule
		\textbf{Média concentração urbana} & \multicolumn{2}{l}{\textbf{População}} & \textbf{Produto Interno Bruto – PIB (bilhões R\$)} & \textbf{Número de empresas} & \textbf{Número de unidades locais} \\ \midrule
		\textbf{Nome}                      & \textbf{Total}   & \textbf{No Brasil}  &                                                   &                             & \\
		Ji-Paraná (RO)                     & 116 610          & 116 610             & 1,686                                             & 2 734                       & 3 082 \\
		Parintins (AM)                     & 102 033          & 102 033             & 0,675                                             & 634                         & 683 \\
		Boa Vista (RR)                     & 298 215          & 298 215             & 4,823                                             & 4 852                       & 5 187 \\
		Bragança (PA)                      & 113 227          & 113 227             & 0,452                                             & 654                         & 686 \\ \bottomrule
	\end{tabular}
	\fonte{\textcite{ibge2016}.}
\end{table}

% ---

% ---
% 3 - Capítulo 3
% ---
% ----------------------------------------------------------
\chapter{Seção}
% ----------------------------------------------------------

Este \textit{template} contém algumas seções criadas na tentativa de facilitar seu uso. No entanto, não há um limite máximo ou mínimo de seção a ser utilizado no trabalho. Cabe a cada autor definir a quantidade que melhor atenda à sua 
necessidade.  
% ---

% ---
% 4 - Conclusão
% ---
%\phantompart
% ----------------------------------------------------------
\chapter{Conclusão}
% ----------------------------------------------------------

As conclusões devem responder às questões da pesquisa, em relação aos objetivos e às hipóteses. Devem ser breves, podendo apresentar recomendações e sugestões para trabalhos futuros.
% ---

% ----------------------------------------------------------
% ELEMENTOS PÓS-TEXTUAIS
% ----------------------------------------------------------
%\postextual
% ----------------------------------------------------------

% ----------------------------------------------------------
% Referências bibliográficas
% ----------------------------------------------------------
\begingroup
    \SingleSpacing
    \setlength\bibitemsep{\baselineskip}
    \printbibliography[title=REFERÊNCIAS]
\endgroup

% ----------------------------------------------------------
% Glossário
% ----------------------------------------------------------
%
% Consulte o manual da classe abntex2 para orientações sobre o glossário.
%
%\glossary

% ----------------------------------------------------------
% Apêndices
% ----------------------------------------------------------

% ---
% Inicia os apêndices
% ---
\begin{apendicesenv}
%   \partapendices*
    % ----------------------------------------------------------
\chapter{Descrição 1}
% ----------------------------------------------------------

Textos elaborados pelo autor, a fim de completar a sua argumentação. Deve ser precedido da palavra APÊNDICE, identificada por letras maiúsculas consecutivas, travessão e pelo respectivo título. Utilizam-se letras maiúsculas dobradas quando esgotadas as letras do alfabeto. 

\end{apendicesenv}
% ---


% ----------------------------------------------------------
% Anexos
% ----------------------------------------------------------

% ---
% Inicia os anexos
% ---
\begin{anexosenv}
%   \partanexos*
    % ----------------------------------------------------------
\chapter{Descrição 2}
% ----------------------------------------------------------

São documentos não elaborados pelo autor que servem como fundamentação (mapas, leis, estatutos). Deve ser precedido da palavra ANEXO, identificada por letras maiúsculas consecutivas, travessão e pelo respectivo título. Utilizam-se letras maiúsculas dobradas quando esgotadas as letras do alfabeto.

\end{anexosenv}

%---------------------------------------------------------------------
% INDICE REMISSIVO
%---------------------------------------------------------------------
%\phantompart
%\printindex
%---------------------------------------------------------------------

\end{document}
